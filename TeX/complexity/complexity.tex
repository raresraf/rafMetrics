

\chapter{Classical Computational Complexity Calculus }


\section{Introduction}
This chapter will present the traditional methodology of calculus in the field of algorithm's computational complexity. The complexity will be expressed as a function $f:\mathbb{N}\longrightarrow\mathbb{R}$, where the function is characterized by the size of the input, while the evaluated value $f(n)$, for a given input size $n$, represents the amount of resources needed in order to compute the result. While the most often analyzed resource is time, expressed in elemental compute operations, the mathematical model is self-reliant for others natures of resources, including space reasoning or hybrid metrics. All the results of this chapter are well known in the literature and they represent the reference standard in calculating algorithm complexity in Computer Science. 

\section{Motivation}
Because of the difficulty of having a rigorous calculus of the complexity, an asymptotic computation approach for an algorithm's complexity offers a valuable insight about the real computational cost without requiring a precisely, flawless calculus.



\section{Family of Bachmann–Landau notations}
The following notations and names will be used for describing the asymptotic behavior of a algorithm's complexity characterized by a function, $f:\mathbb{N}\longrightarrow\mathbb{R}$. \\
We define the set of all complexity calculus $\mathcal{F}= \lbrace f:\mathbb{N}\longrightarrow\mathbb{R} \rbrace$
\\Assume that $n, n_{0}\in\mathbb{N}$. Also, we will consider an arbitrary complexity function $g \in \mathcal{F}$. \\

\begin{definition}
Assume that two continuous and derivable functions $v,w:\mathbb{R}\longrightarrow\mathbb{R}$ are considered to be similar in an \textbf{asymptotic analysis} iff: 
  \[\lim_{x\to\infty} \frac{v(x)}{w(x)} = c \in (-\infty, 0) \cup (0,\infty) \]
\end{definition}

\begin{corollary}
  Consider that if we analyses asymptotic behavior of functions defined over $\mathbb{N}$ the functions $v,w:\mathbb{N}\longrightarrow\mathbb{R}$ are similar iff there exists a function $r:\mathbb{N}\longrightarrow\mathbb{R}$, with $r(x) = \frac{v(x)}{w(x)}\ \forall\ x\in\mathbb{N}$, such that
  \[\lim_{x\to\infty} r(x) = \lim_{x\to\infty} \frac{v(x)}{w(x)} = C \in (-\infty, 0) \cup (0,\infty) \]
\end{corollary}

The following function classes can be therefore defined for any function $g:\mathbb{N}\longrightarrow\mathbb{R}$ that describes a convergent sequence or a divergent sequence with a limit that tends to infinity:
\begin{definition}   
  \textbf{Big Theta}: This set defines the group of mathematical functions similar in magnitude with  $g(n)$ in the study of asymptotic behavior. A set-based description of this group can be expressed as:
  \[\Theta(g(n))= \lbrace f \in \mathcal{F}\ |\ \exists c_{1}, c_{2} \in \mathbb{R}^{*}_{+}, \exists n_{0} \in \mathbb{N}^{*}\ s.t.\ \ c_{1} \cdot g(n) \leq f(n) \leq c_{2} \cdot g(n)\ ,\  \forall n \geq n_{0} \rbrace\]
\end{definition}  

\begin{corollary}
  Another useful method of establishing a Big-Theta relationship, if the composed sequence defined by the ratio of the two functions exists and also $ \lim_{x\to\infty} \frac{f(x)}{g(x)}$ exists, is by applying the sufficient condition:
    \[ f \in \Theta(g(n))\ \Leftrightarrow\ \lim_{x\to\infty} \frac{f(x)}{g(x)} = C \in (-\infty, 0) \cup (0,\infty) \]
\end{corollary}

Remark that  both $f,g$ can define divergent sequence with limits that tend to infinity, while the sequence $r(x) = \frac{f(x)}{g(x)}\ \forall\ x\in\mathbb{N}$ can be convergent and can have a nonzero limit. \\
Iff $ r(x) = \frac{f(x)}{g(x)}\ $ exists, a sufficient condition can be defined using the formal limit definition for an arbitrary sequence if the following condition holds:
  \[\exists \  C \in \mathbb{R}, \ \ C \neq 0 \ s.t. \ \forall \epsilon > 0,\exists n_{0} \in \mathbb{N}\ \ s.t.\ \forall n\geq n_{0}, \ \  |r(n) - C| < \epsilon  \]
  Thus, if $ \exists \  C \in \mathbb{R}, C \neq 0 \  \Rightarrow\  f \in \Theta(g(n))\ $


\begin{definition}   
  \textbf{Big O}: This set defines the group of mathematical functions that are known to have a similar or lower
 asymptotic performance in comparison with  $g(n)$. The set of such functions is defined as it follows:
  \[\mathcal{O}(g(n)) = \lbrace f \in \mathcal{F}\ |\ \exists c \in \mathbb{R}^{*}_{+}, \exists n_{0} \in \mathbb{N}^{*}\ s.t.\  f(n) \leq c \cdot g(n),\  \forall n \geq n_{0} \rbrace\]
  \end{definition}  
  \begin{corollary}
  In order to establish a Big-O relationship, the sufficient condition is that the $ \lim_{x\to\infty} \frac{f(x)}{g(x)}$ exists and it is finite. 
  \[ f \in \mathcal{O}(g(n))\ \Leftrightarrow\ \lim_{x\to\infty} \frac{f(x)}{g(x)} = C \in (-\infty, \infty) \]
  \end{corollary}

  
\begin{definition}   
 \textbf{Big Omega}: This set defines the group of mathematical functions that are known to have a similar or higher asymptotic performance in comparison with  $g(n)$. The set of all function is defined as:
    \[\Omega(g(n)) = \lbrace f \in \mathcal{F}\ |\ \exists c \in \mathbb{R}^{*}_{+}, \exists n_{0} \in \mathbb{N}^{*}\ s.t.\  f(n) \geq c \cdot g(n),\  \forall n \geq n_{0} \rbrace\]
\end{definition}  

  \begin{corollary}
  In order to establish a Big-Omega relationship, a sufficient condition is one of the following:
    \[  \lim_{x\to\infty} \frac{f(x)}{g(x)} = -\infty \Rightarrow f \in \Omega(g(n))\ \]
    \[ \lim_{x\to\infty} \frac{f(x)}{g(x)} = \infty \Rightarrow f \in \Omega(g(n))\  \]
     \[ \lim_{x\to\infty} \frac{f(x)}{g(x)} = C \in (-\infty, 0) \cup (0,\infty) \Rightarrow f \in \Omega(g(n))\ \]
   \end{corollary}
   
\begin{definition}   
 \textbf{Small O}:
  This set defines the group of mathematical functions that are known to have a humble
 asymptotic performance in comparison with  $g(n)$. The set of such functions is defined as it follows:
  \[o(g(n)) = \lbrace f \in \mathcal{F}\ |\ \forall c \in \mathbb{R}^{*}_{+}, \exists n_{0} \in \mathbb{N}^{*}\ s.t.\  f(n) < c \cdot g(n),\  \forall n \geq n_{0} \rbrace\]
  \end{definition}  
  \begin{corollary}
In order to establish a Small-O relationship, the sufficient condition is that the $ \lim_{x\to\infty} \frac{f(x)}{g(x)}$ exists and it is finite. 
  \[ f \in o(g(n))\ \Leftrightarrow\ \lim_{x\to\infty} \frac{f(x)}{g(x)} = 0 \]  
  \end{corollary}
  
\begin{definition}   
 \textbf{Small Omega}:
  This set defines the group of mathematical functions that are known to have a commanding asymptotic performance in comparison with  $g(n)$.
  The set of such functions is defined as it follows:
  \[\omega(g(n)) = \lbrace f \in \mathcal{F}\ |\ \forall c \in \mathbb{R}^{*}_{+}, \exists n_{0} \in \mathbb{N}^{*}\ s.t.\  f(n) > c \cdot g(n),\  \forall n \geq n_{0} \rbrace\]
  \end{definition}  
  \begin{corollary}
In order to establish a Small-Omega relationship, the sufficient condition is that the $ \lim_{x\to\infty} \frac{f(x)}{g(x)}$ exists and one of the following occurs:
  \[  \lim_{x\to\infty} \frac{f(x)}{g(x)} = +\infty \Rightarrow f \in \omega(g(n))\ \]   
  \[  \lim_{x\to\infty} \frac{f(x)}{g(x)} = -\infty \Rightarrow f \in \omega(g(n))\ \]  
  \end{corollary}

\section{Common properties}
We will study few aspects with impact on defining equivalence relations over classes defined in \textit{Bachmann–Landau} notations.
 \hfill\break
 \begin{theorem} Reflexivity property for Big Theta, Big O and Big Omega notations.  \\  $ f \in \Theta(f(n)) $ \\$ f \in \mathcal{O}(f(n)) $ \\$ f \in \Omega(f(n)) $
 \end{theorem}

\begin{theorem} Reflexivity property does not hold for Small O and Small Omega notations/ \\
 $ f \notin \theta(f(n)) $ \\$ f \notin \omega(f(n)) $ 
\end{theorem}

\begin{theorem} Transitivity property for Bachmann–Landau notations.  \\  $ f \in \Theta(g(n)),  g \in \Theta(h(n)) \Rightarrow  f \in \Theta(h(n))$ \\
 $ f \in \mathcal{O}(g(n)),  g \in \mathcal{O}(h(n)) \Rightarrow  f \in \mathcal{O}(h(n))$ \\
 $ f \in \Omega(g(n)),  g \in \Omega(h(n)) \Rightarrow  f \in \Omega(h(n))$ \\
 $ f \in o(g(n)),  g \in o(h(n)) \Rightarrow  f \in o(h(n))$ \\
 $ f \in \omega(g(n)),  g \in \omega(h(n)) \Rightarrow  f \in \omega(h(n))$
\end{theorem} 

\begin{theorem} Symmetry property for Big Theta notation. \\  $ f \in \Theta(g(n)) \Rightarrow g \in \Theta(f(n)) $
\end{theorem} 

\begin{theorem} Transpose symmetry property for Bachmann–Landau notations.  \\  $ f \in \mathcal{O}(g(n)) \Leftrightarrow g \in \Omega(f(n)) $
 \\  $ f \in o(g(n)) \Leftrightarrow g \in \omega(f(n)) $
\end{theorem} 
 
\begin{theorem} Projection property for Big Theta, Big O and Big Omega notations. \\  $ f \in \Theta(g(n)) \Leftrightarrow f \in \mathcal{O}(g(n)), f \in \Omega(g(n)) $
\end{theorem}
 
\section{Addition properties}
Calculus in $Bachmann-Landau$ notations (including Big-O arithmetic) is extremely powerful and obvious when it comes to addition operations. The following relations hold for any correctly defined functions $f, g, h:\mathbb{N}\longrightarrow\mathbb{R}$, where $ h(n) = f(n) + g(n)\  \forall n \in  \mathbb{N} $ and $\exists n_{0} \in \mathbb{N},\ s.t.\ \forall n \in \mathbb{N} \ \Rightarrow f(n) \leq g(n)$:

 
  \begin{lemma} Addition in \textbf{Big Theta}
 \[  h \in \Theta(g)\]
  \end{lemma}
  \begin{lemma} Addition in \textbf{Big O}: 
 \[  h \in \mathcal{O}(g)\]
  \end{lemma}
  \begin{lemma} Addition in \textbf{Big Omega}: 
 \[  h \in \Omega(g)\]
  \end{lemma}
  \begin{lemma} Addition in \textbf{Small O}:
 \[  h \in o(g)\]
  \end{lemma}
  \begin{lemma} Addition in \textbf{Small Omega}:
 \[  h \in \omega(g)\]
  \end{lemma}

  In a relax notation (consider that by any $Bachmann-Landau$ class notation, we denote an arbitrary function part of the class), the following relations can be settled:

  \begin{corollary} Addition in \textbf{Big Theta}: 
  \[  \Theta(f) + \Theta(g) = \Theta(g)\]
  \end{corollary}
  \begin{corollary} Addition in \textbf{Big O}: 
  \[  \mathcal{O}(f) + \mathcal{O}(g) = \mathcal{O}(g)\]
  \end{corollary}  
  \begin{corollary} Addition in \textbf{Big Omega}: 
  \[  \Omega(f) + \Omega(g) = \Omega(g)\]
  \end{corollary}
  \begin{corollary} Addition in \textbf{Small O}:
  \[  o(f) + o(g) = o(g)\]
  \end{corollary}
  \begin{corollary} Addition in \textbf{Small Omega}:
  \[  \omega(f) + \omega(g) = \omega(g)\]
  \end{corollary}
