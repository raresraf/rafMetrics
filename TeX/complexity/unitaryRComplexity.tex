
\chapter{Calculus in {\textit{1-Complexity}} }




\section{Introduction}
This chapter will present a specific set of $rComplexity$ classes, highlighting \textbf{Big \textit{r-}Theta}, \textbf{Big \textit{r-}O} and \textbf{Big \textit{r-}Omega} with unitary parameter (i.e. $r = 1$) by providing useful properties for a more straightforward calculation. The last part of the chapter introduce a new concept for \textit{monotonic, continuous function}: \textbf{normal form representation} and defines a new workflow in $rComplexity$: \textbf{normalized rComplexity calculus}.

\section{Motivation}
Calculus in $rComplexity$ is dependent on the parameter $r \in \mathbb{R}_{+}$, and as a result a large number of operations may require rudimentary \textit{conversions} using relations described in \textit{Common properties} and \textit{Notable properties} chapters. Working with unitary $rComplexity$ classes comes effortless and brings an agile manner of operating ample
calculus using this Complexity Model.

\section{Main notations in 1-Complexity Calculus}
The following notations and names will be used for describing the asymptotic behavior of a algorithm's complexity characterized by a function, $f:\mathbb{N}\longrightarrow\mathbb{R}$ in $1-Complexity$. \\
We define the set of all complexity calculus $\mathcal{F}= \lbrace f:\mathbb{N}\longrightarrow\mathbb{R} \rbrace$
\\Assume that $n, n_{0}\in\mathbb{N}$. Also, we will consider an arbitrary complexity function $g \in \mathcal{F}$. 
The following notations are particularization ($r = 1$) of notations provided in $rComplexity$:
\begin{definition}
\textbf{Big \textit{1-}Theta}: This set defines the group of mathematical functions similar in magnitude with  $g(n)$ in the study of asymptotic behavior. A set-based description of this group can be expressed as:
  \[\begin{split} \Theta_{1}(g(n)) = \lbrace f \in \mathcal{F}\ |\ \forall c_{1}, c_{2} \in \mathbb{R}^{*}_{+} \ s.t.  c_{1} < 1 < c_{2} , \exists n_{0} \in \mathbb{N}^{*}\ \\ s.t.\ \ c_{1} \cdot g(n) \leq f(n) \leq c_{2} \cdot g(n)\ ,\  \forall n \geq n_{0} \rbrace \end{split} \]
\end{definition} 
\begin{definition} 
\textbf{Big \textit{1-}O}: This set defines the group of mathematical functions that are known to have a similar or lower
 asymptotic performance in comparison with  $g(n)$. The set of such functions is defined as it follows:
  \[\mathcal{O}_{1}(g(n)) = \lbrace f \in \mathcal{F}\ |\ \forall c  \in \mathbb{R}^{*}_{+} \ s.t.\  1 < c, \exists n_{0} \in \mathbb{N}^{*}\ s.t.\  f(n) \leq c \cdot g(n),\  \forall n \geq n_{0} \rbrace\]
\end{definition}
\begin{definition}

\textbf{Big \textit{1-}Omega}: This set defines the group of mathematical functions that are known to have a similar or higher asymptotic performance in comparison with  $g(n)$. The set of all function is defined as:
    \[\Omega_{1}(g(n)) = \lbrace f \in \mathcal{F}\ |\ \forall c  \in \mathbb{R}^{*}_{+}\ s.t. \ c < 1, \exists n_{0} \in \mathbb{N}^{*}\ s.t.\  f(n) \geq c \cdot g(n),\  \forall n \geq n_{0} \rbrace\]
\end{definition}
\begin{definition}
\textbf{Small \textit{1-}O}:
  This set defines the group of mathematical functions that are known to have a humble
 asymptotic performance in comparison with  $g(n)$. The set of such functions is defined as it follows:
  \[o_{1}(g(n)) = \lbrace f \in \mathcal{F}\ |\ \forall c \in \mathbb{R}^{*}_{+}, \exists n_{0} \in \mathbb{N}^{*}\ s.t.\  f(n) < c \cdot g(n),\  \forall n \geq n_{0} \rbrace\]
Note that $o_{r}(g(n) = o_{1}(g(n) \forall r\in  \mathbb{R}_{+}$, as Small \textit{1-}O  notation is $r$-independent.
\end{definition}
\begin{definition} \textbf{Small \textit{1-}Omega}:
  This set defines the group of mathematical functions that are known to have a commanding asymptotic performance in comparison with  $g(n)$.
  The set of such functions is defined as it follows:
  \[\omega_{1}(g(n)) = \lbrace f \in \mathcal{F}\ |\ \forall c \in \mathbb{R}^{*}_{+}, \exists n_{0} \in \mathbb{N}^{*}\ s.t.\  f(n) > c \cdot g(n),\  \forall n \geq n_{0} \rbrace\]
    Note that $\omega_{r}(g(n) = \omega_{1}(g(n) \forall r\in  \mathbb{R}_{+}$, as Small \textit{1-}Omega  notation is $r$-independent.
\end{definition}

\section{Asymptotic Analysis}
Calculus in $1-Complexity$ can be performed as well using either limits of sequences or limits of functions. Consider any two complexity functions $f,g:\mathbb{N}_{+}\longrightarrow\mathbb{R}_{+}$.

\begin{theorem} Admittance of a function $f$ in \textbf{Big \textit{1-}Theta} class defined by a function $g$: 
\[ f \in \Theta_{r}(g(n)) \Leftrightarrow \lim_{n\to\infty} \dfrac{f(n)}{g(n)} = 1 \]
\end{theorem} 
\begin{theorem} Admittance of a function $f$ in \textbf{Big \textit{1-}O} class defined by a function $g$: 
\[ f \in \mathcal{O}_{1}(g(n)) \Leftrightarrow \lim_{n\to\infty} \dfrac{f(n)}{g(n)} = l,\ l \in \left[ 0, 1 \right] \]
\end{theorem} 
\begin{theorem} Admittance of a function $f$ in \textbf{Big \textit{1-}Omega} class defined by a function $g$: 
\[ f \in \Omega_{1}(g(n)) \Leftrightarrow \lim_{n\to\infty} \dfrac{f(n)}{g(n)} = l,\ l \in \left[ 1, \infty \right) \]
\end{theorem} 
\begin{theorem} Admittance of a function $f$ in \textbf{Small \textit{1-}O} class defined by a function $g$: 
\[ f \in o_{r}(g(n)) \Leftrightarrow \lim_{n\to\infty} \dfrac{f(n)}{g(n)} = 0 \]
\end{theorem} 
\begin{theorem} Admittance of a function $f$ in \textbf{Small \textit{1-}Omega} class defined by a function $g$: 
 \[ f \in \omega_{r}(g(n)) \Leftrightarrow \lim_{n\to\infty} \dfrac{f(n)}{g(n)} = \infty \]
\end{theorem} 




\section{Common properties}
This chapter will present new implications in terms of reflexivity, transitivity, symmetry and projections properties of calculus in $1-Complexity$.
\begin{theorem} Reflexivity in $1-Complexity$ - Big 1-Omega notation: 
\[ f \in \Theta_{1} \left( f(n) \right)\ \]
\end{theorem} 
\begin{proof} 
   Let $r = 1$ and use Reflexivity property described in Common properties sections for classes in $rComplexity$ calculus.
   \[ f \in \Theta_{r} \left( \frac{1}{r} \cdot f(n) \right)\ \forall r \neq 0 \]
   Therefore for $r = 1$:
 \[ f \in \Theta_{1} \left( f(n) \right)\ \]
\end{proof} 
 
 \begin{theorem} Reflexivity in $1-Complexity$ - Big 1-O notation:
 \[ f \in \mathcal{O}_{1} \left( x \cdot f(n) \right)\ \forall x \geq 1 \]
\end{theorem} 
\begin{proof} 
   Let $r = 1$ and use Reflexivity property for Big \textit{r-}O class in $rComplexity$ calculus.
\end{proof} 

\begin{theorem} Reflexivity in $1-Complexity$ - Big 1-Omega notation:
 \[ f \in \Omega_{1} \left( x \cdot f(n) \right)\ \forall x \leq 1 \]
\end{theorem} 
\begin{proof} 
   Let $r = 1$ and use Reflexivity property for Big \textit{r-}Omega class in $rComplexity$ calculus.
\end{proof} 


The reflexivity does not hold \textbf{either} for Small \textit{1-}o and Small \textit{1-}Omega, as these two sets are equal with the classical sets defined in \textit{Bachmann–Landau notations}.
\begin{theorem} Reflexivity in $1-Complexity$ - Small 1-o notation:
$ f \notin o_{1}(f(n)) $ 
\end{theorem} 
\begin{theorem} Reflexivity in $1-Complexity$ - Small 1-Omega notation:
$ f \notin \omega_{1}(f(n)) $ 
\end{theorem} 

\begin{theorem} Transitivity in $1-Complexity$ - Big 1-Theta notation:  \\  $ f \in \Theta_{1}(g(n)),  g \in \Theta_{1}(h(n)) \Rightarrow  f \in \Theta_{1}(h(n))$   
 \end{theorem} 
\begin{proof} 
   Let $r = 1$ and use Transitivity property for Big \textit{r-}Theta class in $rComplexity$ calculus.
\end{proof} 

\begin{theorem} Transitivity in $1-Complexity$ - Big 1-O notation:
 $ f \in \mathcal{O}_{1}(g(n)),  g \in \mathcal{O}_{1}(h(n)) \Rightarrow  f \in \mathcal{O}_{1}(h(n))$ 
\end{theorem} 
\begin{proof} 
   Let $r = 1$ and use Transitivity property for Big \textit{r-}O class in $rComplexity$ calculus.
\end{proof} 
\begin{theorem} Transitivity in $1-Complexity$ - Big 1-Omega notation:
 $ f \in \Omega_{1}(g(n)),  g \in \Omega_{1}(h(n)) \Rightarrow  f \in \Omega_{1}(h(n))$ 
\end{theorem} 
\begin{proof} 
   Let $r = 1$ and use Transitivity property for Big \textit{r-}Omega class in $rComplexity$ calculus.
\end{proof} 
\begin{theorem} Transitivity in $1-Complexity$ - small notations:
 \[ f \in o_{1}(g(n)),  g \in o_{1}(h(n)) \Rightarrow  f \in o_{1}(h(n)) \]
 \[ f \in \omega_{1}(g(n)),  g \in \omega_{1}(h(n)) \Rightarrow  f \in \omega_{1}(h(n)) \]
\end{theorem} 
\begin{proof} 
   Small \textit{1-}o and Small \textit{1-}Omega are equal with the classical sets defined in \textit{Bachmann–Landau notations} and thus they conserve transitivity properties
\end{proof} 

\begin{theorem} 
 \textbf{Symmetry in $1-Complexity$:}  \\  $ f \in \Theta_{1}(g(n)) \Rightarrow g \in \Theta_{1}(f(n)) $
\end{theorem} 
\begin{proof} 
   Let $r = 1$ and use Symmetry property for Big \textit{r-}Theta class in $rComplexity$ calculus.
\end{proof} 

\begin{theorem} 
 \textbf{Transpose symmetry in $1-Complexity$:}  \\  $ f \in \mathcal{O}_{1}(g(n)) \Leftrightarrow g \in \Omega_{1}(f(n)) $
\end{theorem} 
\begin{proof} 
   Let $r = 1$ and use Transpose symmetry property described in $rComplexity$ calculus.
\end{proof} 
\begin{theorem} Transpose symmetry in $1-Complexity$ in small notations:
 $ f \in o_{1}(g(n)) \Leftrightarrow g \in \omega_{1}(f(n)) $
\end{theorem} 
\begin{proof} 
   Small \textit{1-}o and Small \textit{1-}Omega are equal with the classical sets defined in \textit{Bachmann–Landau notations} and thus they conserve the Transpose symmetry property.
\end{proof} 
 
\begin{theorem} 
  \textbf{Projection in $1-Complexity$:}  \\  $ f \in \Theta_{1}(g(n)) \Leftrightarrow f \in \mathcal{O}_{1}(g(n)), f \in \Omega_{1}(g(n)) $
\end{theorem} 
\begin{proof} 
   Let $r = 1$ and use Projection property described in $rComplexity$ calculus.
\end{proof} 



\section{Addition properties}
Addition properties are obtained by assuming $r = 1$ in the $r$Complexity model.
\begin{theorem} 
Addition properties in \textbf{Big 1-Theta}:  \\
The following relations hold for any correctly defined functions $f, g, f', g', h:\mathbb{N}\longrightarrow\mathbb{R}$, where $ h(n) = f'(n) + g'(n)\  \forall n \in  \mathbb{N} $, where $f',g'$ are two arbitrary functions such that $ f' \in \Theta_{1}(f),  g' \in \Theta_{1}(g) $:  
  \begin{itemize}
  \item \textbf{If} $ \lim_{n\to\infty} \dfrac{f(n)}{g(n)} = 0 \Rightarrow  h \in \Theta_{1}(g) $. 
  \item \textbf{If} $ \lim_{n\to\infty} \dfrac{f(n)}{g(n)} = \infty \Rightarrow  h \in \Theta_{1}(f) $. 
  \item \textbf{If} $ \lim_{n\to\infty} \dfrac{f(n)}{g(n)} = t, \ t \in \mathbb{R}_{+} \Rightarrow  h \in \Theta_{1} \left( f + g \right) $. 
    \end{itemize}
\end{theorem} 

\begin{theorem} 
Addition properties \textbf{Big 1-O}: \\
The following relations hold for any correctly defined functions $f, g, f', g', h:\mathbb{N}\longrightarrow\mathbb{R}$, where $ h(n) = f'(n) + g'(n)\  \forall n \in \mathbb{N} $, where $f',g'$ are two arbitrary functions such that $ f' \in \mathcal{O}_{1}(f),  g' \in \mathcal{O}_{1}(g) $:  
  \begin{itemize}
  \item \textbf{If} $ \lim_{n\to\infty} \dfrac{f(n)}{g(n)} = 0 \Rightarrow  h \in \mathcal{O}_{1}(g) $. 
  \item \textbf{If} $ \lim_{n\to\infty} \dfrac{f(n)}{g(n)} = \infty \Rightarrow  h \in \mathcal{O}_{1}(f) $. 
  \item \textbf{If} $ \lim_{n\to\infty} \dfrac{f(n)}{g(n)} = t, \ t \in \mathbb{R}_{+} \Rightarrow  h \in \mathcal{O}_{1} \left( f + g \right) $.
  \end{itemize}
\end{theorem} 


\begin{theorem} 
Addition properties \textbf{Big 1-Omega}: \\
  The following relations hold for any correctly defined functions $f, g, f', g', h:\mathbb{N}\longrightarrow\mathbb{R}$, where $ h(n) = f'(n) + g'(n)\  \forall n \in \mathbb{N} $, where $f',g'$ are two arbitrary functions such that $ f' \in \Omega_{1}(f),  g' \in \Omega_{1}(g) $:  
  \begin{itemize}
  \item \textbf{If} $ \lim_{n\to\infty} \dfrac{f(n)}{g(n)} = 0 \Rightarrow  h \in \Omega_{1}(g) $.
  \item \textbf{If} $ \lim_{n\to\infty} \dfrac{f(n)}{g(n)} = \infty \Rightarrow  h \in \Omega_{1}(f) $. 
  \item \textbf{If} $ \lim_{n\to\infty} \dfrac{f(n)}{g(n)} = t, \ t \in \mathbb{R}_{+} \Rightarrow  h \in \Omega_{r} \left( f + g \right) $.
  \end{itemize}

\end{theorem} 


In a relax notation (consider that by any  $1-$Complexity class notation, we denote an arbitrary function part of the class), the following relations can be settled:

\begin{lemma}
If $\lim_{n\to\infty} \dfrac{f(n)}{g(n)} = 0$:
\begin{itemize}
  \item \textbf{Big 1-Theta}: 
  \[  \Theta_{1}(f) + \Theta_{1}(g) = \Theta_{1}(g)\]
  \item \textbf{Big 1-O}: 
  \[  \mathcal{O}_{1}(f) + \mathcal{O}_{1}(g) = \mathcal{O}_{1}(g)\]
  \item \textbf{Big 1-Omega}: 
  \[  \Omega_{1}(f) + \Omega_{1}(g) = \Omega_{1}(g)\]
\end{itemize}
\end{lemma}

\begin{lemma}
If $\lim_{n\to\infty} \dfrac{f(n)}{g(n)} = \infty$:
\begin{itemize}
  \item \textbf{Big 1-Theta}: 
  \[  \Theta_{1}(f) + \Theta_{1}(g) = \Theta_{1}(f)\]
  \item \textbf{Big 1-O}: 
  \[  \mathcal{O}_{1}(f) + \mathcal{O}_{1}(g) = \mathcal{O}_{1}(f)\]
  \item \textbf{Big 1-Omega}: 
  \[  \Omega_{1}(f) + \Omega_{1}(g) = \Omega_{1}(f)\]
\end{itemize}
\end{lemma}

\begin{lemma}
If $\lim_{n\to\infty} \dfrac{f(n)}{g(n)} = t, \ t \in \mathbb{R}_{+}$:
\begin{itemize}
  \item \textbf{Big 1-Theta}: 
  \[  \Theta_{1}(f) + \Theta_{1}(g) = \Theta_{1}(f + g)\]
  \item \textbf{Big 1-O}: 
  \[  \mathcal{O}_{1}(f) + \mathcal{O}_{1}(g) = \mathcal{O}_{1}(f + g)\]
  \item \textbf{Big 1-Omega}: 
  \[  \Omega_{1}(f) + \Omega_{1}(g) = \Omega_{1}(f + g)\]
\end{itemize}
\end{lemma}

\section{Normal form functions}
In this section, we will take further steps in simplifying calculus by introducing a new concept: the normal form of a monotonic, continuous function.
\begin{definition}
Let $g:\mathbb{N}^{*}\longrightarrow\mathbb{R}_{+} $ be a monotonic, continuous function. Let the following:
\[ g(n) = \sum_{i=1}^{p} g_{i}(n)\]
be a decomposition, with correctly defined function $g_{i}:\mathbb{N}^{*}\longrightarrow\mathbb{R}_{+} \forall i \in [1, p]$, with the properties:
\begin{itemize}

\item $ \exists \ j, \lim_{n\to\infty} \dfrac{g_{j}(n)}{g(n)} = 1$
\item $\forall i \neq j \  lim_{n\to\infty} \dfrac{g_{j}(n)}{g_{i}(n)} = \infty$ 
\item there is no another decomposition for $g_{j}(n) = \sum_{k=1}^{p'} g_{j_{k}}(n)$ such that $\lim_{n\to\infty} \dfrac{g_{j}(n)}{g_{j_{k}}(n)} = \infty, \forall k \in [1, p']$.
 \end{itemize}
 Then, we call $g_{j}$ to be a function in \textbf{normal form} or in \textbf{atomic form}.
\end{definition}

\begin{remark}
We will refer to a monotonic, continuous function $g$ that is in normal form using the notation $g_{1}(n)$ .
\end{remark}


\begin{remark}
Working in $1$-Complexity with functions in normal form will involve converting an arbitrary monotonic, continuous function into an atomic representation, given by the normal form $g_{1}(n)$, by applying the decomposition presented in the definition above.This step involve few additional overhead in calculus, but overall the conversions are accessible and simplifies a lot the progress of complexity detection for various algorithms. Nonetheless, it provides powerful benchmarkings solutions for comparing algorithms' efficiency. 
\end{remark}

\section{Normalized rComplexity Calculus}
\begin{definition}
We will denote by: 
\[ f(n) \in \Theta_{1}(g_{1}(n)) \]
a function f that is in the Big 1-Theta $1-$Complexity class defined by the function $g_{1}$ and $g_{1}$ is in normal form. Working with normal form functions for the class characterization in $1-$Complexity Calculus will be named \textbf{normalized rComplexity calculus}.
\end{definition}

\begin{remark}
Working in $r$-Complexity with the required conversions such that $r = 1$ and the functions defining complexity classes is in normal form $(g = g_{1})$ is the most simple
calculus model while working with $r$-Complexity classes.
\end{remark}