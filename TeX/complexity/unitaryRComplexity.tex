
\chapter{Calculus in {\textit{1-Complexity}} }




\section{Introduction}
This chapter will present a specific set of $rComplexity$ classes, highlighting \textbf{Big \textit{r-}Theta}, \textbf{Big \textit{r-}O} and \textbf{Big \textit{r-}Omega} with unitary parameter (i.e. $r = 1$) by providing useful properties for a more straightforward calculation. 

\section{Motivation}
Calculus in $rComplexity$ is dependent on the parameter $r \in \mathbb{R}_{+}$, and as a result a large number of operations may require rudimentary \textit{conversions} using relations described in \textit{Common properties} and \textit{Notable properties} chapters. Working with unitary $rComplexity$ classes comes effortless and brings an agile manner of operating ample
calculus using this Complexity Model.

\section{Main notations in 1-Complexity Calculus}
The following notations and names will be used for describing the asymptotic behavior of a algorithm's complexity characterized by a function, $f:\mathbb{N}\longrightarrow\mathbb{R}$ in $1-Complexity$. \\
We define the set of all complexity calculus $\mathcal{F}= \lbrace f:\mathbb{N}\longrightarrow\mathbb{R} \rbrace$
\\Assume that $n, n_{0}\in\mathbb{N}$. Also, we will consider an arbitrary complexity function $g \in \mathcal{F}$. 
The following notations are particularization ($r = 1$) of notations provided in $rComplexity$:
\begin{itemize}
  \item \textbf{Big \textit{1-}Theta}: This set defines the group of mathematical functions similar in magnitude with  $g(n)$ in the study of asymptotic behavior. A set-based description of this group can be expressed as:
  \[\begin{split} \Theta_{1}(g(n)) = \lbrace f \in \mathcal{F}\ |\ \forall c_{1}, c_{2} \in \mathbb{R}^{*}_{+} \ s.t.  c_{1} < 1 < c_{2} , \exists n_{0} \in \mathbb{N}^{*}\ \\ s.t.\ \ c_{1} \cdot g(n) \leq f(n) \leq c_{2} \cdot g(n)\ ,\  \forall n \geq n_{0} \rbrace \end{split} \]
  
  \item \textbf{Big \textit{1-}O}: This set defines the group of mathematical functions that are known to have a similar or lower
 asymptotic performance in comparison with  $g(n)$. The set of such functions is defined as it follows:
  \[\mathcal{O}_{1}(g(n)) = \lbrace f \in \mathcal{F}\ |\ \forall c  \in \mathbb{R}^{*}_{+} \ s.t.\  1 < c, \exists n_{0} \in \mathbb{N}^{*}\ s.t.\  f(n) \leq c \cdot g(n),\  \forall n \geq n_{0} \rbrace\]
  
  \item \textbf{Big \textit{1-}Omega}: This set defines the group of mathematical functions that are known to have a similar or higher asymptotic performance in comparison with  $g(n)$. The set of all function is defined as:
    \[\Omega_{1}(g(n)) = \lbrace f \in \mathcal{F}\ |\ \forall c  \in \mathbb{R}^{*}_{+}\ s.t. \ c < 1, \exists n_{0} \in \mathbb{N}^{*}\ s.t.\  f(n) \geq c \cdot g(n),\  \forall n \geq n_{0} \rbrace\]

  \item \textbf{Small \textit{1-}O}:
  This set defines the group of mathematical functions that are known to have a humble
 asymptotic performance in comparison with  $g(n)$. The set of such functions is defined as it follows:
  \[o_{1}(g(n)) = \lbrace f \in \mathcal{F}\ |\ \forall c \in \mathbb{R}^{*}_{+}, \exists n_{0} \in \mathbb{N}^{*}\ s.t.\  f(n) < c \cdot g(n),\  \forall n \geq n_{0} \rbrace\]
Note that $o_{r}(g(n) = o_{1}(g(n) \forall r\in  \mathbb{R}_{+}$, as Small \textit{1-}O  notation is $r$-independent.
  
  \item \textbf{Small \textit{1-}Omega}:
  This set defines the group of mathematical functions that are known to have a commanding asymptotic performance in comparison with  $g(n)$.
  The set of such functions is defined as it follows:
  \[\omega_{1}(g(n)) = \lbrace f \in \mathcal{F}\ |\ \forall c \in \mathbb{R}^{*}_{+}, \exists n_{0} \in \mathbb{N}^{*}\ s.t.\  f(n) > c \cdot g(n),\  \forall n \geq n_{0} \rbrace\]
    Note that $\omega_{r}(g(n) = \omega_{1}(g(n) \forall r\in  \mathbb{R}_{+}$, as Small \textit{1-}Omega  notation is $r$-independent.

\end{itemize}

\section{Asymptotic Analysis}
Calculus in $1-Complexity$ can be performed as well using either limits of sequences or limits of functions. Consider any two complexity functions $f,g:\mathbb{N}_{+}\longrightarrow\mathbb{R}_{+}$.

\begin{itemize}
  \item \textbf{Big \textit{1-}Theta}: 
  \[ f \in \Theta_{r}(g(n)) \Leftrightarrow \lim_{n\to\infty} \dfrac{f(n)}{g(n)} = 1 \]

  \item \textbf{Big \textit{1-}O}: 
    \[ f \in \mathcal{O}_{1}(g(n)) \Leftrightarrow \lim_{n\to\infty} \dfrac{f(n)}{g(n)} = l,\ l \in \left[ 0, 1 \right] \]
  
  \item \textbf{Big \textit{1-}Omega}: 
      \[ f \in \Omega_{1}(g(n)) \Leftrightarrow \lim_{n\to\infty} \dfrac{f(n)}{g(n)} = l,\ l \in \left[ 1, \infty \right) \]

  \item \textbf{Small \textit{1-}O}:
    \[ f \in o_{r}(g(n)) \Leftrightarrow \lim_{n\to\infty} \dfrac{f(n)}{g(n)} = 0 \]

  \item \textbf{Small \textit{1-}Omega}:
    \[ f \in \omega_{r}(g(n)) \Leftrightarrow \lim_{n\to\infty} \dfrac{f(n)}{g(n)} = \infty \]

\end{itemize}



\section{Common properties}
This chapter will present new implications in terms of reflexivity, transitivity, symmetry and projections properties of calculus in $1-Complexity$.
 \hfill\break
 \textbf{Reflexivity in $1-Complexity$:}  \[ f \in \Theta_{1} \left( f(n) \right)\ \]
   \\\textit{Proof}:
   Let $r = 1$ and use Reflexivity property described in Common properties sections for classes in $rComplexity$ calculus.
   \[ f \in \Theta_{r} \left( \frac{1}{r} \cdot f(n) \right)\ \forall r \neq 0 \]
   Therefore for $r = 1$:
 \[ f \in \Theta_{1} \left( f(n) \right)\ \]
 \qedsymbol
 
 
 \[ f \in \mathcal{O}_{1} \left( x \cdot f(n) \right)\ \forall x \geq 1 \]
   \\\textit{Proof}:
   Let $r = 1$ and use Reflexivity property for Big \textit{r-}O class in $rComplexity$ calculus.
 \qedsymbol

 \[ f \in \Omega_{1} \left( x \cdot f(n) \right)\ \forall x \leq 1 \]
   \\\textit{Proof}:
   Let $r = 1$ and use Reflexivity property for Big \textit{r-}Omega class in $rComplexity$ calculus.
 \qedsymbol


The reflexivity does not hold \textbf{either} for Small \textit{1-}o and Small \textit{1-}Omega, as these two sets are equal with the classical sets defined in \textit{Bachmann–Landau notations}
 
$ f \notin o_{1}(f(n)) $ \\$ f \notin \omega_{1}(f(n)) $ 
 \hfill\break

 \textbf{Transitivity in $1-Complexity$:}  \\  $ f \in \Theta_{1}(g(n)),  g \in \Theta_{1}(h(n)) \Rightarrow  f \in \Theta_{1}(h(n))$   
    \\\textit{Proof}:
   Let $r = 1$ and use Transitivity property for Big \textit{r-}Theta class in $rComplexity$ calculus.
 \qedsymbol
 \\ \\
 $ f \in \mathcal{O}_{1}(g(n)),  g \in \mathcal{O}_{1}(h(n)) \Rightarrow  f \in \mathcal{O}_{1}(h(n))$ 
     \\\textit{Proof}:
   Let $r = 1$ and use Transitivity property for Big \textit{r-}O class in $rComplexity$ calculus.
 \qedsymbol
 \\ \\
 $ f \in \Omega_{1}(g(n)),  g \in \Omega_{1}(h(n)) \Rightarrow  f \in \Omega_{1}(h(n))$ 
     \\\textit{Proof}:
   Let $r = 1$ and use Transitivity property for Big \textit{r-}Omega class in $rComplexity$ calculus.
 \qedsymbol
 \\ \\
 $ f \in o_{1}(g(n)),  g \in o_{1}(h(n)) \Rightarrow  f \in o_{1}(h(n))$ 
     \\\textit{Proof}:
   Small \textit{1-}o and Small \textit{1-}Omega are equal with the classical sets defined in \textit{Bachmann–Landau notations} and thus they conserve transitivity properties
 \qedsymbol
 \\ \\
 $ f \in \omega_{1}(g(n)),  g \in \omega_{1}(h(n)) \Rightarrow  f \in \omega_{1}(h(n))$
     \\\textit{Proof}:
   Small \textit{1-}o and Small \textit{1-}Omega are equal with the classical sets defined in \textit{Bachmann–Landau notations} and thus they conserve transitivity properties
 \qedsymbol
  \hfill\break

 \textbf{Symmetry in $1-Complexity$:}  \\  $ f \in \Theta_{1}(g(n)) \Rightarrow g \in \Theta_{1}(f(n)) $
     \\\textit{Proof}:
   Let $r = 1$ and use Symmetry property for Big \textit{r-}Theta class in $rComplexity$ calculus.
 \qedsymbol
 \hfill\break

 \textbf{Transpose symmetry in $1-Complexity$:}  \\  $ f \in \mathcal{O}_{1}(g(n)) \Leftrightarrow g \in \Omega_{1}(f(n)) $
      \\\textit{Proof}:
   Let $r = 1$ and use Transpose symmetry property described in $rComplexity$ calculus.
 \qedsymbol
 \\ \\
 \\  $ f \in o_{1}(g(n)) \Leftrightarrow g \in \omega_{1}(f(n)) $
      \\\textit{Proof}:
   Small \textit{1-}o and Small \textit{1-}Omega are equal with the classical sets defined in \textit{Bachmann–Landau notations} and thus they conserve the Transpose symmetry property.
 \qedsymbol

 \hfill\break
 
  \textbf{Projection in $1-Complexity$:}  \\  $ f \in \Theta_{1}(g(n)) \Leftrightarrow f \in \mathcal{O}_{1}(g(n)), f \in \Omega_{1}(g(n)) $
      \\\textit{Proof}:
   Let $r = 1$ and use Projection property described in $rComplexity$ calculus.
 \qedsymbol
 \\ \\
 \hfill\break

 
 
 
\section{Addition properties}
Addition properties are obtained by assuming $r = 1$ in the $r$Complexity model.
\begin{itemize}
  \item \textbf{Big 1-Theta}:  \\
The following relations hold for any correctly defined functions $f, g, f', g', h:\mathbb{N}\longrightarrow\mathbb{R}$, where $ h(n) = f'(n) + g'(n)\  \forall n \in  \mathbb{N} $, where $f',g'$ are two arbitrary functions such that $ f' \in \Theta_{1}(f),  g' \in \Theta_{1}(g) $:  
  \\ \\
  \textbf{If} $ \lim_{n\to\infty} \dfrac{f(n)}{g(n)} = 0 \Rightarrow  h \in \Theta_{1}(g) $. \\

  \textbf{If} $ \lim_{n\to\infty} \dfrac{f(n)}{g(n)} = \infty \Rightarrow  h \in \Theta_{1}(f) $. \\

  \textbf{If} $ \lim_{n\to\infty} \dfrac{f(n)}{g(n)} = t, \ t \in \mathbb{R}_{+} \Rightarrow  h \in \Theta_{1} \left( f + g \right) $. \\

  \item \textbf{Big 1-O}: \\
The following relations hold for any correctly defined functions $f, g, f', g', h:\mathbb{N}\longrightarrow\mathbb{R}$, where $ h(n) = f'(n) + g'(n)\  \forall n \in \mathbb{N} $, where $f',g'$ are two arbitrary functions such that $ f' \in \mathcal{O}_{1}(f),  g' \in \mathcal{O}_{1}(g) $:  
  \\ \\

  \textbf{If} $ \lim_{n\to\infty} \dfrac{f(n)}{g(n)} = 0 \Rightarrow  h \in \mathcal{O}_{1}(g) $. \\
  \textbf{If} $ \lim_{n\to\infty} \dfrac{f(n)}{g(n)} = \infty \Rightarrow  h \in \mathcal{O}_{1}(f) $. \\
  \textbf{If} $ \lim_{n\to\infty} \dfrac{f(n)}{g(n)} = t, \ t \in \mathbb{R}_{+} \Rightarrow  h \in \mathcal{O}_{1} \left( f + g \right) $. \\


  \item \textbf{Big 1-Omega}: 
  The following relations hold for any correctly defined functions $f, g, f', g', h:\mathbb{N}\longrightarrow\mathbb{R}$, where $ h(n) = f'(n) + g'(n)\  \forall n \in \mathbb{N} $, where $f',g'$ are two arbitrary functions such that $ f' \in \Omega_{1}(f),  g' \in \Omega_{1}(g) $:  
  \\ \\
  \textbf{If} $ \lim_{n\to\infty} \dfrac{f(n)}{g(n)} = 0 \Rightarrow  h \in \Omega_{1}(g) $. \\

  \textbf{If} $ \lim_{n\to\infty} \dfrac{f(n)}{g(n)} = \infty \Rightarrow  h \in \Omega_{1}(f) $. \\
  \textbf{If} $ \lim_{n\to\infty} \dfrac{f(n)}{g(n)} = t, \ t \in \mathbb{R}_{+} \Rightarrow  h \in \Omega_{r} \left( f + g \right) $. \\

\end{itemize}


In a relax notation (consider that by any  $1-$Complexity class notation, we denote an arbitrary function part of the class), the following relations can be settled:


If $\lim_{n\to\infty} \dfrac{f(n)}{g(n)} = 0$:
\begin{itemize}
  \item \textbf{Big 1-Theta}: 
  \[  \Theta_{1}(f) + \Theta_{1}(g) = \Theta_{1}(g)\]
  \item \textbf{Big 1-O}: 
  \[  \mathcal{O}_{1}(f) + \mathcal{O}_{1}(g) = \mathcal{O}_{1}(g)\]
  \item \textbf{Big 1-Omega}: 
  \[  \Omega_{1}(f) + \Omega_{1}(g) = \Omega_{1}(g)\]
\end{itemize}


If $\lim_{n\to\infty} \dfrac{f(n)}{g(n)} = \infty$:
\begin{itemize}
  \item \textbf{Big 1-Theta}: 
  \[  \Theta_{1}(f) + \Theta_{1}(g) = \Theta_{1}(f)\]
  \item \textbf{Big 1-O}: 
  \[  \mathcal{O}_{1}(f) + \mathcal{O}_{1}(g) = \mathcal{O}_{1}(f)\]
  \item \textbf{Big 1-Omega}: 
  \[  \Omega_{1}(f) + \Omega_{1}(g) = \Omega_{1}(f)\]
\end{itemize}

If $\lim_{n\to\infty} \dfrac{f(n)}{g(n)} = t, \ t \in \mathbb{R}_{+}$:
\begin{itemize}
  \item \textbf{Big 1-Theta}: 
  \[  \Theta_{1}(f) + \Theta_{1}(g) = \Theta_{1}(f + g)\]
  \item \textbf{Big 1-O}: 
  \[  \mathcal{O}_{1}(f) + \mathcal{O}_{1}(g) = \mathcal{O}_{1}(f + g)\]
  \item \textbf{Big 1-Omega}: 
  \[  \Omega_{1}(f) + \Omega_{1}(g) = \Omega_{1}(f + g)\]
\end{itemize}

