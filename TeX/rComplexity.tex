
\chapter{A refined complexity calculus model:\newline \textbf{\textit{r-Complexity}} }




\section{Introduction}
This chapter will present a new approach in the field of algorithm's computational complexity. Similar to the conventional asymptotic notations proposed in the literature,  \textit{rComplexity} will be expressed as a function $f:\mathbb{N}\longrightarrow\mathbb{R}$, where the function is characterized by the size of the input, while the evaluated value $f(n)$, for a given input size $n$, represents the amount of resources needed in order to compute the result. 
This new calculus model aims to produce new asymptotic notations that offer better complexity feedback for similar algorithms, providing subtle insights even for algorithms that are part of the same conventional complexity class $\Theta(g(n))$ denoted by an arbitrary function $g:\mathbb{N}\longrightarrow\mathbb{R}$, in the definition of \textit{Bachmann–Landau} notations .
\section{Motivation}
The classical complexity calculus model is a long-established, verified metric of evaluating an algorithm's performance and a valuable measurement in estimating feasibility of computing a considerable algorithm. However, the model has a shortfall in making discrepancy between similar algorithms with two similar complexity functions, $v,w:\mathbb{N}\longrightarrow\mathbb{R}$, with $v(n),w(n) \in \Theta(g(n))$. In order to highlight the lack of distinction, suppose two algorithms $Alg1$ and $Alg2$ that solve exactly the same problem, with the following complexity functions:
$f_{1},f_{2}:\mathbb{N}\longrightarrow\mathbb{R}$ where $f_{1} = x \cdot f_{2}$, with $x \in \mathbb{R}_{+}, \ x > 1$.
If \[ f_{2} \in \Theta(g(n)) \Rightarrow \exists c_{1}, c_{2} \in \mathbb{R}^{*}_{+}, \exists n_{0} \in \mathbb{N}^{*}\ s.t.\ \ c_{1} \cdot g(n) \leq f_{2}(n) \leq c_{2} \cdot g(n)\ ,\  \forall n \geq n_{0} \]
Therefore, $f_{1} \in \Theta(g(n))$, as there exists $ c_{1}^{'}, c_{2}^{'} \in \mathbb{R}^{*}_{+},  n_{0}^{'} \in \mathbb{N}^{*}$ such that $ c_{1}^{'} =  x \cdot c_{1} $, $ c_{2}^{'} =  x \cdot c_{2} $ and $n_{0}^{'} = n_{0}$ and  $\forall n \geq n_{0}^{'} $ :  $c_{1}^{'} \cdot g(n) \leq f_{1}(n) \leq c_{2}^{'} \cdot g(n)$.
\\ \\ 
Remark that even if $f_{1} > f_{2}$, both complexity functions are part of the same complexity class. This observation implies that two algorithms, whose complexity functions can differ by a constant, assume equal with $2020^{2020}$, are part of the same complexity class, even if the actual run-time might differ by over $6676$ orders of magnitude. For comparison, only a $30$ magnitude order between the two complexity functions $f_{1} = 10^{30} \cdot f_{2}$, signify
that if for a given input $n$, if $Alg2$ ends execution in 1 \textit{attosecond} ($0.000000001$ part of a nanosecond), then $Alg1$ is expected to end execution in about 3 \textit{millenniums}. Despite of the colossal difference in time, classical complexity model is not perceptive between algorithms whose complexity functions differs only through constants. \\ \\ 
\textbf{\textit{rComplexity}} calculus aims to clarify this issue by taking into deep analysis the preeminent constants that can state major improvements in an algorithm's complexity and can have tremendous effect over total execution time. 


\section{Adjusting the Bachmann–Landau notations for rComplexity Calculus}
The following notations and names will be used for describing the asymptotic behavior of a algorithm's complexity characterized by a function, $f:\mathbb{N}\longrightarrow\mathbb{R}$. \\
We define the set of all complexity calculus $\mathcal{F}= \lbrace f:\mathbb{N}\longrightarrow\mathbb{R} \rbrace$
\\Assume that $n, n_{0}\in\mathbb{N}$. Also, we will consider an arbitrary complexity function $g \in \mathcal{F}$. 
Acknowledge the following notations $\forall r \neq 0$:
\begin{itemize}
  \item \textbf{Big \textit{r-}Theta}: This set defines the group of mathematical functions similar in magnitude with  $g(n)$ in the study of asymptotic behavior. A set-based description of this group can be expressed as:
  \[\begin{split} \Theta_{r}(g(n)) = \lbrace f \in \mathcal{F}\ |\ \forall c_{1}, c_{2} \in \mathbb{R}^{*}_{+} \ s.t.  c_{1}< r < c_{2} , \exists n_{0} \in \mathbb{N}^{*}\ \\ s.t.\ \ c_{1} \cdot g(n) \leq f(n) \leq c_{2} \cdot g(n)\ ,\  \forall n \geq n_{0} \rbrace \end{split} \]
  
  \item \textbf{Big \textit{r-}O}: This set defines the group of mathematical functions that are known to have a similar or lower
 asymptotic performance in comparison with  $g(n)$. The set of such functions is defined as it follows:
  \[\mathcal{O}_{r}(g(n)) = \lbrace f \in \mathcal{F}\ |\ \forall c  \in \mathbb{R}^{*}_{+} \ s.t.\  r<c, \exists n_{0} \in \mathbb{N}^{*}\ s.t.\  f(n) \leq c \cdot g(n),\  \forall n \geq n_{0} \rbrace\]
  
  \item \textbf{Big \textit{r-}Omega}: This set defines the group of mathematical functions that are known to have a similar or higher asymptotic performance in comparison with  $g(n)$. The set of all function is defined as:
    \[\Omega_{r}(g(n)) = \lbrace f \in \mathcal{F}\ |\ \forall c  \in \mathbb{R}^{*}_{+}\ s.t. \ c < r, \exists n_{0} \in \mathbb{N}^{*}\ s.t.\  f(n) \geq c \cdot g(n),\  \forall n \geq n_{0} \rbrace\]

  \item \textbf{Small \textit{r-}O}:
  This set defines the group of mathematical functions that are known to have a humble
 asymptotic performance in comparison with  $g(n)$. The set of such functions is defined as it follows:
  \[o_{r}(g(n)) = \lbrace f \in \mathcal{F}\ |\ \forall c \in \mathbb{R}^{*}_{+}, \exists n_{0} \in \mathbb{N}^{*}\ s.t.\  f(n) < c \cdot g(n),\  \forall n \geq n_{0} \rbrace\]
  This set is defined for symmetry of the model and it is equal with the set defined by \textbf{Small O} notation in \textit{Bachmann–Landau notations}, as the definition is independent on $r$.
  
  \item \textbf{Small \textit{r-}Omega}:
  This set defines the group of mathematical functions that are known to have a commanding asymptotic performance in comparison with  $g(n)$.
  The set of such functions is defined as it follows:
  \[\omega_{r}(g(n)) = \lbrace f \in \mathcal{F}\ |\ \forall c \in \mathbb{R}^{*}_{+}, \exists n_{0} \in \mathbb{N}^{*}\ s.t.\  f(n) > c \cdot g(n),\  \forall n \geq n_{0} \rbrace\]
    This set is defined for symmetry of the model and it is equal with the set defined by \textbf{Small Omega} notation in \textit{Bachmann–Landau notations}, as the definition is independent on $r$.
\end{itemize}

\section{Asymptotic Analysis}
Calculus in $rComplexity$ can be performed either using limits of sequences or limits of functions. Consider any two complexity functions $f,g:\mathbb{N}_{+}\longrightarrow\mathbb{R}_{+}$.

\begin{itemize}
  \item \textbf{Big \textit{r-}Theta}: 
  \[ f \in \Theta_{r}(g(n)) \Leftrightarrow \lim_{n\to\infty} \dfrac{f(n)}{g(n)} = r \]

  \item \textbf{Big \textit{r-}O}: 
    \[ f \in \mathcal{O}_{r}(g(n)) \Leftrightarrow \lim_{n\to\infty} \dfrac{f(n)}{g(n)} = l,\ l \in \left[ 0, r \right] \]
  
  \item \textbf{Big \textit{r-}Omega}: 
      \[ f \in \Omega_{r}(g(n)) \Leftrightarrow \lim_{n\to\infty} \dfrac{f(n)}{g(n)} = l,\ l \in \left[ r, \infty \right) \]

  \item \textbf{Small \textit{r-}O}:
    \[ f \in o_{r}(g(n)) \Leftrightarrow \lim_{n\to\infty} \dfrac{f(n)}{g(n)} = 0 \]

  \item \textbf{Small \textit{r-}Omega}:
    \[ f \in \omega_{r}(g(n)) \Leftrightarrow \lim_{n\to\infty} \dfrac{f(n)}{g(n)} = \infty \]
    
\end{itemize}



\section{Common properties}
This chapter will present new implications in terms of reflexivity, transitivity, symmetry and projections compared with conventional \textit{Bachmann–Landau notations}
 \hfill\break
 \textbf{Reflexivity in $rComplexity$:}  \[ f \in \Theta_{r} \left( \frac{1}{r} \cdot f(n) \right)\ \forall r \neq 0 \]
   \\\textit{Proof}:
    Using the definition of $ \Theta_{r}(f(n))$
    \[\begin{split} \Theta_{r}(f(n)) = \lbrace f' \in \mathcal{F}\ |\ \forall c_{1}, c_{2} \in \mathbb{R}^{*}_{+} \ s.t.  c_{1}< r < c_{2} , \exists n_{0} \in \mathbb{N}^{*}\ \\ s.t.\ \ c_{1} \cdot f(n) \leq f'(n) \leq c_{2} \cdot f(n)\ ,\  \forall n \geq n_{0} \rbrace \end{split} \]
    Using substitution $ f(n) \longleftarrow \frac{1}{r} \cdot f(n)$
    \[\begin{split} \Theta_{r} \left( \frac{1}{r} \cdot f(n) \right) = \lbrace f' \in \mathcal{F}\ |\ \forall c_{1}, c_{2} \in \mathbb{R}^{*}_{+} \ s.t.  c_{1}< r < c_{2} , \exists n_{0} \in \mathbb{N}^{*}\ \\ s.t.\ \frac{1}{r} \cdot \ c_{1} \cdot f(n) \leq  f'(n) \leq \frac{1}{r} \cdot c_{2} \cdot f(n)\ ,\  \forall n \geq n_{0} \rbrace \end{split} \]
    By choosing $c_{1}' = \frac{1}{r} \cdot c_{1}, c_{2}' = \frac{1}{r} \cdot c_{2}$
   \[\begin{split} \Theta_{r} \left( \frac{1}{r} \cdot f(n) \right) = \lbrace f' \in \mathcal{F}\ |\ \forall c_{1}', c_{2}' \in \mathbb{R}^{*}_{+} \ s.t.  c_{1}'< 1 < c_{2}' , \exists n_{0} \in \mathbb{N}^{*}\ \\ s.t.\  \ c_{1}' \cdot f(n) \leq  f'(n) \leq c_{2}' \cdot f(n)\ ,\  \forall n \geq n_{0} \rbrace \end{split} \]
	For any $x \in \mathbb{R}^{*}$,  $\forall c_{1}, c_{2}\ \ s.t.  c_{1} \leq 1 \leq c_{2}$, we have $ x \cdot c_{1} \leq x \leq x \cdot c_{2} $. \\
   	Thus, if $f:\mathbb{N}\longrightarrow\mathbb{R}$, $\forall c_{1}, c_{2}\ \  s.t.  c_{1} \leq 1 \leq c_{2}$, we have $f(n) \cdot c_{1} \leq f(n) \leq f(n) \cdot c_{2}\ \ \forall n \in \mathbb{N}^{*}$ or $f(n) \cdot c_{1} \geq f(n) \geq f(n) \cdot c_{2}\ \ \forall n \in \mathbb{N}^{*}$\\
   	Therefore:
   	\[\forall c_{1}', c_{2}' \in \mathbb{R}^{*}_{+} \ s.t.  c_{1}'< 1 < c_{2}' , \exists n_{0} = 1  s.t.\  \ c_{1}' \cdot f(n) \leq  f(n) \leq c_{2}' \cdot f(n)\ ,\  \forall n \geq n_{0}=1 \Rightarrow \]
   	\[ f \in \Theta_{r} \left( \frac{1}{r} \cdot f(n) \right)\ \forall r \neq 0 \]
\qedsymbol
 
 
 \[ f \in \mathcal{O}_{r} \left( x \cdot f(n) \right)\ \forall x \geq \dfrac{1}{r} \]
   \\\textit{Proof}:
    Using the definition of $ \mathcal{O}_{r}(f(n))$
  \[\mathcal{O}_{r}(f(n)) = \lbrace f' \in \mathcal{F}\ |\ \forall c  \in \mathbb{R}^{*}_{+} \ s.t.\  r<c, \exists n_{0} \in \mathbb{N}^{*}\ s.t.\  f'(n) \leq c \cdot f(n),\  \forall n \geq n_{0} \rbrace\]
    Using substitution $ f(n) \longleftarrow x \cdot f(n), \forall x \geq \dfrac{1}{r}, \forall r \neq 0$
  \[\mathcal{O}_{r}(x \cdot f(n)) = \lbrace f' \in \mathcal{F}\ |\ \forall c  \in \mathbb{R}^{*}_{+} \ s.t.\  r<c, \exists n_{0} \in \mathbb{N}^{*}\ s.t.\  f'(n) \leq c \cdot x \cdot f(n),\  \forall n \geq n_{0} \rbrace\]
If $r<c$ and $x \geq \dfrac{1}{r}$, then $c \cdot x \geq 1, \forall r,c,x\ \ r \neq 0$
   	Thus, if $f:\mathbb{N}\longrightarrow\mathbb{R}$, we have $ f(n) \leq 1 \cdot f(n) \leq  c \cdot x\cdot f(n) \ \ \forall n \in \mathbb{N}^{*}$. 
   	Therefore:
   	\[\forall x \geq \dfrac{1}{r}, \ \forall c \in \mathbb{R}^{*}_{+} \ s.t.\  r<c , \exists n_{0} = 1  s.t.\  \  f(n) \leq  c \cdot x\cdot f(n) \ \ \forall n \in \mathbb{N}^{*} ,\  \forall n \geq n_{0}=1 \Rightarrow \]
 \[ f \in \mathcal{O}_{r} \left( x \cdot f(n) \right)\ \forall x \geq \dfrac{1}{r} \]
\qedsymbol

 \[ f \in \Omega_{r} \left( x \cdot f(n) \right)\ \forall x \leq \dfrac{1}{r} \]
    Using the definition of $\Omega_{r}(f(n))$
  \[\Omega_{r}(f(n)) = \lbrace f' \in \mathcal{F}\ |\ \forall c  \in \mathbb{R}^{*}_{+} \ s.t.\  c<r, \exists n_{0} \in \mathbb{N}^{*}\ s.t.\  f'(n) \geq c \cdot f(n),\  \forall n \geq n_{0} \rbrace\]
    Using substitution $ f(n) \longleftarrow x \cdot f(n), \forall x \leq \dfrac{1}{r}, \forall r \neq 0$
  \[\Omega_{r}(x \cdot f(n)) = \lbrace f' \in \mathcal{F}\ |\ \forall c  \in \mathbb{R}^{*}_{+} \ s.t.\  c<r, \exists n_{0} \in \mathbb{N}^{*}\ s.t.\  f'(n) \geq x \cdot c \cdot f(n),\  \forall n \geq n_{0} \rbrace\]
If $r>c$ and $x \leq \dfrac{1}{r}$, then $c \cdot x \leq 1, \forall r,c,x\ \ r \neq 0$
   	Therefore:
   	\[\forall x \leq \dfrac{1}{r}, \ \forall c \in \mathbb{R}^{*}_{+} \ s.t.\  c<r , \exists n_{0} = 1  s.t.\  \  f(n) \geq  c \cdot x\cdot f(n) \ \ \forall n \in \mathbb{N}^{*} ,\  \forall n \geq n_{0}=1 \Rightarrow \]
 \[ f \in \Omega_{r} \left( x \cdot f(n) \right)\ \forall x \leq \dfrac{1}{r} \]
\qedsymbol

The reflexivity does not hold for Small \textit{r-}o and Small \textit{r-}Omega, as these two sets are equal with the classical sets defined in \textit{Bachmann–Landau notations}:
 
$ f \notin o_{r}(f(n)) $ \\$ f \notin \omega_{r}(f(n)) $ 
 \hfill\break

 \textbf{Transitivity in $rComplexity$:}  \\  $ f \in \Theta_{r}(g(n)),  g \in \Theta_{r'}(h(n)) \Rightarrow  f \in \Theta_{r \cdot r'}(h(n))$    \\\textit{Proof}:

	If $ f \in \Theta_{r}(g(n)) \Rightarrow \forall c_{1}, c_{2} \in \mathbb{R}^{*}_{+} \ s.t.  c_{1}< r < c_{2} , \exists n_{0} \in \mathbb{N}^{*}\ \\ s.t.\ \ c_{1} \cdot g(n) \leq f(n) \leq c_{2} \cdot g(n)\ ,\  \forall n \geq n_{0} $ \\
	If $ g \in \Theta_{r'}(h(n)) \Rightarrow \forall c_{1}, c_{2} \in \mathbb{R}^{*}_{+} \ s.t.  c_{1} < r' < c_{2} , \exists n_{0}' \in \mathbb{N}^{*}\ \\ s.t.\ \ c_{1} \cdot h(n) \leq g(n) \leq c_{2} \cdot h(n)\ ,\  \forall n \geq n_{0}' $ \\
	Thus, if $ f \in \Theta_{r}(g(n))$ and $ g \in \Theta_{r'}(h(n))$ \\ $\Rightarrow \forall c_{1}, c_{2}, c_{1}', c_{2}' \in \mathbb{R}^{*}_{+} \ s.t.  c_{1} < r < c_{2}, c_{1}' < r' < c_{2}' , \exists n''_{0}=max(n_{0}, n'_{0}) \in \mathbb{N}^{*}\ \\ s.t.\ \ c_{1}' \cdot c_{1} \cdot h(n) \leq c_{1} \cdot g(n) \leq f(n) \leq c_{2} \cdot g(n) \leq c_{2}' \cdot c_{2} \cdot h(n)\ ,\  \forall n \geq n''_{0} $ \\
	Let $c_{1}'' = c_{1} \cdot c_{1}' , c_{2}'' = c_{2} \cdot c_{2}'$. \\
	Then, $c_{1}'' < r \cdot r' < c_{2}''$ \\ 
	Therefore, $\forall c_{1}'', c_{2}'' \in \mathbb{R}^{*}_{+} \ s.t.  c_{1}'' < r \cdot r' < c_{2}'' \ \ \exists n''_{0}=max(n_{0}, n'_{0}) \in \mathbb{N}^{*}\ \\ s.t.\ \ c_{1}'' \cdot h(n) \leq f(n) \leq c_{2}'' \cdot h(n)\ ,\  \forall n \geq n''_{0} \Rightarrow f \in \Theta_{r \cdot r'}(h(n))$ \\

\qedsymbol

All other Transitivity properties hold as well in $rComplexity$ Calculus. The proof is similar with the above.

 $ f \in \mathcal{O}_{r}(g(n)),  g \in \mathcal{O}_{r}(h(n)) \Rightarrow  f \in \mathcal{O}_{r \cdot r'}(h(n))$ \\
 $ f \in \Omega_{r}(g(n)),  g \in \Omega_{r}(h(n)) \Rightarrow  f \in \Omega_{r \cdot r'}(h(n))$ \\
 $ f \in o_{r}(g(n)),  g \in o_{r'}(h(n)) \Rightarrow  f \in o_{r \cdot r'}(h(n))$ \\
 $ f \in \omega_{r}(g(n)),  g \in \omega_{r'}(h(n)) \Rightarrow  f \in \omega_{r \cdot r'}(h(n))$
  \hfill\break

 \textbf{Symmetry in $rComplexity$:}  \\  $ f \in \Theta_{r}(g(n)) \Rightarrow g \in \Theta_{\frac{1}{r}}(f(n)) $
\\\textit{Proof}:
$ f \in \Theta_{r}(g(n)) \Rightarrow \forall c_{1}, c_{2} \in \mathbb{R}^{*}_{+} \ s.t.  c_{1}< r < c_{2} , \exists n_{0} \in \mathbb{N}^{*}\ \\ s.t.\ \ c_{1} \cdot g(n) \leq f(n) \leq c_{2} \cdot g(n)\ ,\  \forall n \geq n_{0} $
\\ Using the substitution $c_{1}' = \dfrac{c_{1}}{r}, c_{2}' = \dfrac{c_{2}}{r} \Rightarrow \forall c_{1}', c_{2}' \in \mathbb{R}^{*}_{+} \ s.t.  c_{1} < 1 < c_{2} , \exists n_{0} \in \mathbb{N}^{*}\ \\ s.t.\ \ c_{1}' \cdot g(n) \leq \dfrac{1}{r} \cdot f(n) \leq c_{2}' \cdot g(n)\ ,\  \forall n \geq n_{0} $
\\ The previous inequality can be re-written as: 
\[\begin{cases} g(n) \leq \dfrac{1}{c_{1}'} \cdot \dfrac{1}{r} \cdot f(n) \\ g(n) \geq \dfrac{1}{c_{2}'} \cdot \dfrac{1}{r} \cdot f(n)
 \end{cases}\]
 \\ Using notation $c_{1}'' = \dfrac{1}{c_{2}'}, c_{2}'' = \dfrac{1}{c_{1}'}$ the inequality becomes:  \\
 $\forall c_{1},'' c_{2}'' \in \mathbb{R}^{*}_{+} \ s.t.  c_{1}'' < r < c_{2}'' , \exists n_{0} \in \mathbb{N}^{*}\ $
 \[ {c_{1}''} \cdot \dfrac{1}{r} \cdot f(n) \leq g(n) \leq {c_{2}'} \cdot \dfrac{1}{r} \cdot f(n)\ \ \forall n \geq n_{0} \]
 Thus, using the definition of $ \Theta_{r}(f(n))$, $ f \in \Theta_{r}(g(n)) \Rightarrow g \in \Theta_{\frac{1}{r}}(f(n)) $.
\qedsymbol


 \hfill\break

 \textbf{Transpose symmetry in $rComplexity$:}  \\  $ f \in \mathcal{O}_{r}(g(n)) \Leftrightarrow g \in \Omega_{\frac{1}{r}}(f(n)) $
 \\  $ f \in o_{r}(g(n)) \Leftrightarrow g \in \omega_{\frac{1}{r}}(f(n)) $
 \hfill\break
 
  \textbf{Projection in $rComplexity$:}  \\  $ f \in \Theta_{r}(g(n)) \Leftrightarrow f \in \mathcal{O}_{r}(g(n)), f \in \Omega_{r}(g(n)) $
 \hfill\break

 
 
 
\section{Addition properties}
TODO


\section{Other notable properties}
TODO
